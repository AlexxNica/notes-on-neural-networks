\documentclass[12pt]{report}

\usepackage{hyperref}

\newcommand{\link}[2]{\href{#1}{#2}}


\begin{document}

\title{Notes on neural networks --- CIFAR material}
\author{Michael Nielsen\thanks{Email: mn@michaelnielsen.org}$^{,}$\thanks{Web: http://michaelnielsen.org/ddi}}

\maketitle

\chapter{Introduction}

\textbf{Working notes, by Michael Nielsen:} These are rough working
notes, written as part of my study of neural networks, especially work
on CIFAR.  Note that they really are \emph{rough}, and I've made no
attempt to clean them up, nor do I plan to.  They contain
misunderstandings, misinterpretations, omissions, and outright errors.
As such, I don't advise others to read the notes, and certainly not to
rely on them!

\chapter{Papers}

\section{KSH configuration}


Note that the bias initialization parameter initB was not set anywhere
in the KSH configuration.  That means it defaults to 0.

\textbf{Layer 1}
\begin{itemize}
\item Convolutional
\item 3 channels.  
\item 32 filters
\item Padding of 2.  Pads the images on the outside with a 2-pixel border.
\item Stride length of 1.
\item Filter size is 5 by 5.
\item initW=0.0001.  The initial standard deviation.  I'm surprised by how
  low this is --- much lower than I would have guessed.  I wonder if
  there's any benefit to increasing it?
\item partialSum=4.  No idea what this means.  The docs don't really say.
\item sharedBiases=1.  According to the docs, "indicates that the biases
  of every filter in this layer should be shared amongst all
  applications of that filter."  This is a little unclear.  Does it
  mean that all filters have the same bias?
\item Fully linear layer.
\end{itemize}

\textbf{Layer 2}

+ Pooling layer
+ Uses maxpooling
+ start=0.  Where to start pooling.  This is just the default, which
  is to start pooling where you'd expect (the top left).
+ sizeX=3.  Pool 3 x 3 regions.
+ stride=2.  The stride length.
+ outputsX=0.  This is an unimportant default; if not equal to 0 the
  output would only cover part of the image.
+ channels=32.  Presumably to correspond to the filters in the last
  layer.
+ neuron=relu

Layer 3

+ Convolutional layer
+ 32 filters output, 32 channels input.
+ 5 by 5 filters.
+ Stride length of 1
+ Initial weight SD = 0.01
+ Rectified linear units
+ sharedBiases=1
+ partialSum=4

Layer 4:
+ Pooling layer
+ Average pooling
+ 3 x 3 pooling windows
+ Stride length 2

Layer 5:
+ COnvolutional layer.
+ 32 input channels, 64 output filters
+ 5 x 5 filters
+ Padding by 2 pixel border
+ Stride length of 1
+ Initial weight SD = 0.01
+ Recitified linear units
+ sharedBiases=1
+ partialSum=4

Layer 6:
+ Pooling layer, 64 input channels, 64 outputs
+ Average pooling
+ 3 x 3 poling windows.
+ Stride length 2

Layer 7:
+ Fully connected layer
+ 64 outputs
+ Initial weight SD = 0.1
+ Rectified linear units

Layer 8:
+ Fully connected layer
+ 10 outputs
+ Initial weight SD = 0.1.
+ Linear neurons

Layer 9:
+ SOftmax layer, producing 10 outputs

Cost function: logistic regression on the Softmax outputs.


Learning parameters

Layer 1 (first convolutional layer):
+ Weight learning rate: 0.001
+ Bias learning rate: 0.002
+ Weight and bias momentum: 0.9
+ Weight decay 0.004.  Note there is no bias decay.

Note that in the docs Krizhevsky explicitly gives the update rule:

w' = (weight momentum) * w - (weight decay) * (weight learning rate) * w
+ (weight learning rate) * gradient

The bias rule is the same, but there is no bias weight decay.


Layer 3 (second convolutional layer)

Same as layer 1.

\textbf{Layer 5 (third convolutional layer):} Same as layer 1.


\textbf{Layer 7 (first fully connected layer):} Learning rates as for
convolutional layers, and weight decay of 0.03

\textbf{Layer 8 (final layer)}: Same as first fully connected layer.

Krizhevsky notes that rescaling the overall cost function has the
effect of changing the effective overall learning rate.


\section{Wan et al (2013) -- ``Regularization of Neural Networks using
  DropConnect}

\subsection{Summary of the main points} 

\begin{itemize}
\item Dropout means randomly deleting half the neurons
when training.  

\item DropConnect means randomly deleting half the connections when
  training.  

\item Note that the output is defined as the \emph{average} output
  over the sampled networks, not the full network.  

\item There is a nice linear algebraic way of representing DropConnect
  and Dropout, using Hadamard products, which no doubt helps in
  implementations.  

\item In actual fact, they don't literally implement DropConnect.
  Rather, they analyse what the distribution of weighted sums would
  be, and approximate by a Gaussian, before sampling.  I don't see why
  they do this (it may be faster), but in some sense we can use this
  as a definition.  I'd probably prefer just to sample.  No idea why
  they don't.

\item They claim that the regularization is greatly helped by using
  small mini-batches, ideally mini-batch size $1$ (online learning).

\item The code is available.  They used cuda-convnet for convolutional
  and softmax steps.  The DropConnect implementation is a bit
  convoluted --- worth reading about the problems they had, though.
  It certainly seems worth storing the masks as bits or ints, not
  floats.

\item Used mini-batch SGD with momentum on batches of 128 images, and
  momentum fixed at 0.9.  Not clear how this relates to the above
  comments about online learning.  They augment the dataset (cropping,
  flipping, scaling and rotation); train 5 independent network with
  random permutuations; manually decrease the learning rate using a
  validation set; train using Dropout, DropConnect, or neither.  Use
  1,000 samples.  Use a bias learning rate twice the weight learning
  rate.  Weights are N(0, 0.1) for fully connected layers, and N(0,
  0.01) for convolutional layers.

\item The learning schedule is fascinating.  ``We report three numbers
  of epochs, such as 600-400-200 to define our schedule.  We multiply
  the initial rate by 1 for the first such number of epochs.  Then we
  use a multipler of 0.5 for the second number of epochs followed by
  0.1 again for this second number of epochs.  The third number of
  epochs is used for multipliers of 0.05, 0.01, 0.005, and 0.001 in
  that order, after which point we report our results.  We determine
  the epochs to use for our schedule using a validation set to look
  for plateaus in the loss function, at which point we move to the
  next multiplier.''

\item CIFAR-10: Subtract per-pixel mean computed over the training
  set.  Then use KSH's 3-layer convolutional net.  Follow by 64-unit
  fully connected layer to which DropConnect etc may be applied.  No
  data augmentation. 150-0-0 epochs, a single model, with an initial
  learning rate of 0.0001, and KSH's weight decay (0.995, I believe).
  DropConnect prevents overfitting a little better than Dropout.

\item CIFAR-10: More advanced results.  Using 2 conv layers, 2 locally
  connected layers, per KSH.  128 neuron fully connected layer with
  ReLU activations between softmax and feature extractor.  Images are
  cropped to 24 by 24 to get more data.  Initial learning rate: 0.001,
  and train for 700-300-50 epochs with KSH's weight decay.  Model
  voting helps a \emph{lot}, getting error rate 9.41 percent.  This
  can be improved to 9.32 percent by using 12 networks.

\end{itemize}

\subsection{Other notes}

``When training with Dropout, a randomly selected subset of
activations are set to zero within each layer.  DropConnect instead
sets a randomly selected subset of weights within the network to
zero.''

As with Dropout, DropConnect is essentially a method of
regularization, to prevent the network from overtraining.  ``In
practice, using these [regularization] techniques when training big
networks gives superior test performance to smaller networks trained
without regularization.''

On Dropout: ``Although a full understanding of its mechanism is
elusive, the intuition is that it prevents the network weights from
collaborating with one another to memorize the training examples.''

``Like Dropout, [DropConnect] is suitable for fully connected layers only.''

I don't really see why.  Does something go wrong if we apply it to a
convolutional net?  I don't see why something analogous couldn't be
done.

We can rewrite Dropout as $a \rightarrow \sigma(m \odot (wa+b))$,
where $\odot$ is the Hadamard product, and $m$ is a binary mask
vector, chosen according to an appropriate Bernoulli distribution.  A
similarly nice expression can be obtained for DropConnect.  (This
seems likely to help in implementations.)

\textbf{Architecture:} A CNN, followed by a DropConnect layer,
followed by a SoftMax, and a cross-entropy loss.

Note that the output value can be viewed as the result of sampling a
large number of different (though overlapping) neural networks.

``A key component to successfully training with DropConnect is the
selection of a different mask for each training example.  Selecting a
single mask for a subset of training examples, such as a mini-batch of
128 examples, does not regularize the model enough in practice.''

They define the output as the result of averaging over all
DropConnected networks.  Note that this seems likely to be superior to
using the entire network (i.e., with no weights deleted).

They do some odd things involving Gaussian moment matching to sample.
I don't see \emph{why} they need to do this, I must admit.  But it
does give a reasonably nice way of approximating the network.
Alternately, one could view it as the definition of DropConnect.


\textbf{Q: How do Dropout and DropConnect fare in a sparse network?}
My guess is that they'll show very interesting behaviour.



\chapter{Queue}

LeCun 2013.

Snoek, Larochelle, Adams.

Model voting.

Hinton Dropout paper.

Bengio's dropout paper.

ReLU.

\end{document}