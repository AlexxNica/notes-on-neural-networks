\documentclass[12pt]{article}

\usepackage{hyperref}

\newcommand{\link}[2]{\href{#1}{#2}}


\begin{document}

\title{Notes on neural networks}
\author{Michael Nielsen\thanks{Email: mn@michaelnielsen.org}$^{,}$\thanks{Web: http://michaelnielsen.org/ddi}}

\maketitle

\textbf{Working notes, by Michael Nielsen:} These are rough working
notes, written as part of my study of neural networks.  Note that they
really are \emph{rough}, and I've made no attempt to clean them up,
nor do I plan to.  They contain misunderstandings, misinterpretations,
omissions, and outright errors.  As such, I don't advise others to
read the notes, and certainly not to rely on them!

\textbf{Core questions:} There is a practical, narrow question: what
are the most significant results about deep learning and neural
networks?  And then there is the broader question: how to build an
artificial intelligence?  My reading will address both questions.

\textbf{Recurrent neural networks (RNN):} According to Wikipedia, RNNs
have achieved the best results to date on handwriting recognition.  An
obvious question is: what are the respective advantages of RNNs and
feedforward networks?  Are there important problems for which one or
the other is preferable?  Why?  What I've read about these questions
is opaque.

\textbf{Williams and Zipser (1989) - a gradient-based learning method
  for recurrent neural networks:}
\link{http://scholar.google.ca/scholar?cluster=1352799553544912946\&hl=en\&as\_sdt=0,5}{(link)}
  
Claims that feedforward networks don't have the "ability to store
information for later use".  It'd be nice to understand what that
means.  Obviously there's a trivial sense in which feedforward
networks can store information based on training data.  

Claims that backprop requires lots of memory when used with large
amounts of training data.  I don't believe this, except in the trivial
sense that it may take a lot of memory to store all the training data.
Otherwise, we can compute gradients
training-instance-by-training-instance, and sum the results, which is
not especially memory intensive.  (Of course, one may have a huge
network which requires a lot of memory to story.  But that's a
separate issue.)

Their model of recurrent neural networks is interesting.  Basically,
we have a set of neurons, each with an output.  And we have a set of
inputs to the network.  There is a weight between every pair of
neurons, and from each input to each neuron.  To compute a neuron's
output at time $t+1$ we compute the weighted sum of the inputs and the
outputs at time $t$, and apply the appropriate nonlinear function
(sigmoid, or whatever).  Note that in order for this description to
make sense we must specify the behaviour of the external inputs over
time.  We can incorporate a bias by having an external input which is
always $1$.
  
So a recurrent neural network is just like a feedforward network, with
a weight constraint: the weights in each layer are the same, over and
over again.  Also, the inputs must be input to every layer in the
network.
  
Williams and Zipser take as their supervised training task the goal of
getting neuron outputs to match certain desired training values at
certain times.  For instance, you could define a two-neuron network
that will \emph{eventually} produce the XOR of the inputs.
  
They define the total error to be the sum over squares of the errors
in individual neuron outputs.  And we can then do ordinary gradient
descent with that error function.  They derive a simple dynamical
system to describe how to improve the weights in the network using
gradient descent.

The above algorithm assumes that the weights in the network remain
constant for all time.  Williams and Zipser then modify the learning
algorithm, allowing it to change the weights at each step.  The idea
is simply to compute the total error at any \emph{given} time, and
then to use gradient descent with that error function to update the
weights.  (Similar to online learning in feedforward networks.)
  
Williams and Zipser describe a method of \emph{teacher-forcing},
modifying the neural network by replacing the output of certain
neurons by the \emph{desired} output, for training purposes in later
steps.
  
Unfortunately, it is still unclear to me \emph{why} one would wish to
use recurrent neural networks.  Williams and Zipser describe a number
of examples, but they don't seem compelling.

The algorithm in which the weights can change seems non-physiological
--- it verges on being an unmotivated statistical model.  (I doubt
that the weights in the brain swing around wildly, but I'll bet that
the weights found by this algorithm can swing around wildly.)  The
algorithm in which the weights are fixed seems more biological.
  
Note that Williams and Zipser \emph{do not} offer any analysis of
running time for their algorithms, or an understanding of when it is
likely to work well, and when it is not.  It's very much in the
empirical let's-see-how-this-works style adopted through much of the
neural networks literature.
  
Summing up: the recurrent neural network works by, at each step,
computing the sigmoid function of the weighted sum of the inputs and
the previous step's outputs.  Training means specifying a set of
desired outputs at particular times, and adapting the weights at each
time-step.  Training works by specifying an error function at any
given time step, computing the gradient, and updating the weights
appropriately.

\textbf{Compiling to neural networks:} Can we create compilers which
translate programs written in a conventional programming language into
a neural network?  I'd be especially interested in seeing how this
works for AI workhorses such as Prolog.  What could we learn from such
a procedure?  (1) Perhaps we could figure out how to link up multiple
neural modules, with one or more of the modules coming from the
compiler? (2) Maybe we could use a learning technique to further
improve the performance of the compiled network.  Googling doesn't
reveal a whole lot, although I did find a paper by
\link{http://scholar.google.ca/scholar?cluster=10518384657895134615\&hl=en\&as\_sdt=0,5}{Thrun}
where he discusses decompiling, i.e., extracting rules from a neural
network.  Thrun uses a technique he calls validity-interval analysis,
basically propagating intervals for inputs and outputs forwards and
backwards through a network.

\textbf{Softmax function:} Suppose $q_j$ is some set of values.  Then
  we define the softmax function by:
\begin{eqnarray}
p_j \equiv \exp(q_j)/\sum_k \exp(q_j).
\end{eqnarray}
This is a probability distribution, which preserves the order of the
original values.  You can, for example, take the softmax in the final
layer of a neural network, taking the weighted sum of inputs as the
$q_j$ values, and then applying the softmax.  The output from the
network can then be interpreted as a probability distribution.

\textbf{Baldi and Hornik, 1989 - characterizes linear autoencoders}:
\link{http://scholar.google.ca/scholar?cluster=11637720331851320383&hl=en&as_sdt=0,5}{(link)}
We have a three-layer network, and the output is related to the input
by $x \rightarrow ABx$, where $B$ describes the first layer of
weights, and $A$ the second layer.  The goal is to find weight
matrices $A$ and $B$ to minimize:
\begin{eqnarray}
\sum_x \|x-ABx\|^2.
\end{eqnarray}
The challenge is that the hidden layer has a \emph{smaller} number $h$
of neurons than the input layer (which is, of course, of the same size
as the output layer)\footnote{It's not quite clear to me what $h$
  should parameterize.  I'll use it to parameterize the number of
  dimensions in the vector space representing outputs from the hidden
  units.  It seems likely that it'd be better to write $2^h$, but I'll
  ignore that.}.  Let me try an attack on this without reading the
paper.  That sum above is just:
\begin{eqnarray}
\mbox{tr}((I-AB)^2 \Sigma),
\end{eqnarray}
where $\Sigma \equiv \sum_x x x^T$.  To minimize this what we want to
do is obvious (and easily proven): we'll choose $A$ and $B$ so that
$AB$ is a $h$-dimensional projector onto the span of the eignenvectors
of $\Sigma$ with the $h$ largest eigenvalues.  Let $P(\Sigma, h)$
denote such a projector, so:
\begin{eqnarray}
AB = P(\Sigma, h).
\end{eqnarray}
We can easily characterize such $A$ and $B$.  $A$ should take the
space $P(\Sigma, h)$ into the space spanned by the outputs from the
hidden units, and $B$ should then undo that transformation.  There is
an orthogonal freedom inbetween time, and a possible freedom in
$P(\Sigma, h)$.  This completely characterizes $A$ and $B$.

Summing up, in a linear neural network, \emph{a linear autoencoder is
  just doing principal components analysis}.  So \emph{a non-linear
  autoencoder can be thought of as a non-linear generalization of
  PCA}.  That's a useful fact to remember.  Examination of the
remainder of the paper suggests that these are the key facts.

\textbf{Thinking geometrically:} Suppose we're asked to tell the
difference between pictures of a human face, and pictures of a
giraffe.  We can represent the pictures as points $x$ in a very
high-dimensional space.  And so our task is to divide that space up
into two parts: one is classified as giraffe, the other as human face.
(Maybe it should be three parts: the thrid part would be: neither face
nor giraffe).  And so what we really want is algorithms for dividing
up that space.  In some sense we're interested in understanding the
space of all such algorithms. 

It'd be interesting to lay out all the different curlicues to thinking
in this way: the opportunities, and the pitfalls.  There are at least
three broad approaches: (1) the \emph{pure geometric approach}, based
on finding mathematical structures to divide the space; (2) the
\emph{biological approach}, where we try to figure out how we do it;
and (3) the \emph{kludge approach}, where we simply try lots of ideas,
and pile them up on top of one another.  That's a pretty rough
division, but seems like a good starting point for thought.  My bet is
that progress comes from playing these ideas off against one another.

\textbf{Finding linear approximations (PCA):} It'll be useful to
review PCA here.  Suppose we have a set of data points $x$ in some
high-dimensional (vector) space.  Then we'd like to find a
$k$-dimensional projector $P$ such that the following error function
is minimized:
\begin{eqnarray}
\sum_x \| x-Px \|^2.
\end{eqnarray}
This error can be rewritten as $\mbox{tr}((I-P)\Sigma)$, where $\Sigma
\equiv \sum_x x x^T$.  And so we simply choose $P$ to project onto the
eigenvectors of $\Sigma$ with the $k$ largest eigenvalues.  The
\emph{principal components} are the eigenvectors of $\Sigma$, in order
of decreasing eigenvalue.  (There may, of course, be some ambiguity
when $\Sigma$ is degenerate).

Practically speaking, suppose we have a billion images, each of which
can be regarded as a vector in a 100,000-dimensional space.  We can
reduce to (say) a 100-dimensional space.  This gets rid of much of the
irrelevant structure, and hopefully leaves a structure that is useful
for comparing images.

\textbf{We don't seem to have much theory of what it means to
  generalize:} We have all these techniques based on
parameter-fitting.  But we have a paucity of strong underlying
theoretical ideas.

\textbf{Deep learning requires nonlinear neurons:} Put another way,
deep learning with linear neurons doesn't help.  Via linear embedding
it's equivalent to a single hidden layer whose size is just the
minimal size of any of the original hidden layers.  So there is
absolutely no advantage to doing deep learning with linear neurons.

\textbf{Tenenbaum, de Silva and Langford, 2000:}
\link{http://scholar.google.ca/scholar?cluster=14602426245887619907&hl=en&as_sdt=0,5}{(link)}
There's a lot I can learn from this paper, so I'll take a lot of
notes.  Some of those notes I'll factor out.  

They mention a technique called multidimensional scaling (MDS), which
I hadn't heard of.  The general idea seems to be that we have a lot of
items, and we know some ``dissimilarities'' between items.  The goal
is to find a metric space embedding of those items so that the
distances are roughly equal to the dissimilarities.

A sample problem: we have a 4096-dimensional space, corresponding to
64 by 64 pixel images.  A (nonlinear) subspace of this corresponds to
images we'd recognize as faces.  How can we characterize this
subspace?  

This is just one possible mathematical formalization of the problem.
In practice, things are more complex.  Our classification will be
fuzzy.  We'll have all kinds of extra contextual information: maybe
we've got an external hint; maybe we can see a nose; maybe the colour
is wrong, but we see enough to suspect it's false colour.  All these
kinds of things are clearly important in how we actually see.  In
other words, we don't just have an algorithm for face detection.  We
have a million related algorithms, and they all affect how well face
detection works.  In some sense you don't solve one problem perfectly.
You solve a network of problems imperfectly --- and then use those
results to improve your performance on the original problem.  It's a
kind of \emph{learning network}.  In a sense this is what a deep
neural network does: it builds up gradually more complicated features.

The algorithm they describe is very simple.  Very roughly (this
certainly contains mistakes): the idea seems to be to take all your
data points and to compute distances between them.  We assume that
when the distances are small, the points are neighbours.  Construct a
graph in which neighbouring points are connected.  Then geodesic
distance is found (approximated) by finding the shortest distance in
the graph.  We then embed the graph in a space of the chosen
dimensionality.  Nice!  Simple, probably pretty easy to implement, and
I expect it lets us find a lot of structure.

It's worth thinking about what the input and output are.  The input to
Iso-map is just a data set --- maybe it's a set of images of a face,
maybe it's a set of words, whatever.  This data lives in a very
high-dimensional space.  What we do is we find an embedding in a much
lower dimensional space --- say, 2-dimensional.  In other words, we're
constructing new features, based on the original features.

\textbf{There are $10^6$ optic nerves and $30,000$ auditory nerves:}
I'm not quite sure what to make of this.  Presumably it means that we
process something like $30$ times as much optical information as
auditory.  I wonder how pixellated the information is?  

\textbf{What happens when we augment the features, with PCA?}  Let's
suppose we start off with 3 features, $x, y, z$.  Then we add $x^2$
and $y^2$ as new features.  Certain subsets of the original space that
weren't linearly approximable \emph{will be} in the new feature space.
This seems like a potentially powerful technique.  What can it be used
to do?  What are its limits?

\textbf{Tenenbaum, Kemp, Griffiths, and Goodman, 2011:}
\link{http://scholar.google.ca/scholar?cluster=2667398573353002097&hl=en&as_sdt=0,5}{(link)}
A review of a particular approach to inductive learning.  They want to
combine Bayesian learning with complex ways of representing knowledge.

Claims that there is strong evidence that children can learn to
generalize their use of words from just a few examples.  This suggests
that there must be some pretty clever underlying patterns to how we
generalize.  ``A massive mismatch looms between information coming in
through our senses and the outputs of cognition''.  

Claims that we humans do reason (implicitly) in Bayesian ways about a
number of things.  Mostly omits the evidence that we \emph{don't} in
some important ways.  This omission bugs me.  They \emph{do} mention
the fact that our conscious assessements of probability tend to be
terrible, which is pleasing.  With that said, I'm not certain about
this --- I just have the strong impression that there are well-known
instances where we certainly don't reason in a Bayesian way.  It'd be
good to have references.

``The biggest remaining obstacle is to understand how structured
symbolic knowledge can be represented in neural circuits.''
Interesting.  I've often wondered exactly this.  They make the
followup comment: ``Connectionist models sidestep these challenges by
denying that brains actually encode such rich knowledge''.  That seems
too strong to me, but there is some truth to it: the connectionists
seem less interested than one might suppose in this question, perhaps
believing that its solution should be deferred.

How would one go about solving this problem?  Actually, what would a
solution / better statement of the problem even look like?  Maybe we
could encode entry-relationships?  In particular, let us suppose we
want to encode $X Y Z$ where $X$ and $Z$ are entities, and $Y$ is the
relationship.  One way of encoding this would be to have a neural
network with nodes for each entity and for each relationship.  We'd
try to design the network so that the only relationships which are
active would be those which are true, given the active entities.

\textbf{Olshausen and Field (1996):} Presents a method for finding
low-complexity representations of natural images, in terms of atomic
images --- which they call ``sparse codes'' --- which are localized,
oriented, and scale-sensitive.  These are found using an unsupervised
learning algorithm with a bias toward good quality, low-complexity
representations.  The codes seem to be quite similar to the receptive
fields found in the human visual system.

The \emph{receptive field} for a cell in the retina is the volume of
space (roughly, a cone) which can stimulate that cell to fire.  Nearby
cells can have overlapping (or nearby) receptive fields.  Other cells
in the visual cortex also have receptive fields, but they may be more
complex, since the light has already been filtered through one or more
levels of processing.  

The paper claims that the receptive fields in the primary visual
cortex are: (a) spatially localized; (b) oriented; and (c) can
distinguish structure at different scales.

There is then a question: so what are those receptive fields?  In a
way, we can view this as being the question: to what type of images do
different cells in our primary visual cortex respond?  Answering that
question seems like a good start for understanding any higher-level
image processing.  It's the question: what are the atoms of image
processing?  Or perhaps a better way is to think of them as the
molecules of image processing, since they're one level up from the
pixel level.

They develop an unsupervised learning algorithm which, trained on
natural data, can find receptive fields that are spatially localized,
oriented, and can distinguish structure at different scales.

Olshausen and Field want to decompose an image as:
\begin{eqnarray}
  I(x,y) = \sum_j a_j \phi_j(x,y).
\end{eqnarray}
The idea is that the $\phi_j$ form a (possibly overcomplete) basis for
the space of images.  They want to choose the $\phi_j$ which ``results
in the coefficient values being as statistically independent as
possible over an ensemble of natural images''.  In some sense, the
different $a_j$ would be ``telling us different things'' about the
image.  They also want the coefficient values to be sparse, favouring
simple representations over more complex.

O \& F try to search for a suitable set of $\phi_j$s by introducing an
error function:
\begin{eqnarray}
  E = -\mbox{[preserve information]}-\lambda\mbox{[sparseness of } a_j {]}.
\end{eqnarray}
This error is \emph{for a single image}.  The first term is just the
$l_2$ error, i.e., (minus) the quadratic distance between the image
and its representation.  The sparseness term is just a nonlinear
function of the $a_j$ coefficients, quantifying how sparse they are.

The idea is to do online learning with this error function, presenting
it with natural images, and gradually minimizing the error.  (I see
later in the article that it was actually batch learning using
conjugate gradient descent.  It appears that some kind of average
error is being computed.)  The result will be an overcomplete basis
set that favours sparse decompositions of images.

The ``sparsification'' idea is a very interesting one.  Basically,
it's a way of trying to force a kind of Occam's razor into the system.
It's a bit like autoencoders, forcing a simple explanation of complex
data.

O \& F note that wavelets have been used to find sparse codes
previously.

\textbf{Restricted Boltzmann machines:} The idea is not to learn a
function, but rather to learn a probability distribution.  There are
two layers of neurons: a visible layer, and a hidden layer.  All
visible units are connected to all hidden units.  The energy of a
given configuration is just:
\begin{eqnarray}
  E(v, h) = -\sum_i a_i v_i-\sum_j b_j h_j-\sum_{ij} w_{ij} v_i h_j \\
  & = & -a \cdot v-b\cdot h -v^T W h,
\end{eqnarray}
where $a$ are the biases for the visible units, $b$ are the biases for
the hidden units, and $W$ is the weight matrix.  The distribution is
just the standard Boltzmann distribution, at some fixed temperature.
Apparently it can be shown that:
\begin{eqnarray}
  p(v_i = 1 | h) = \sigma( a_i + (Wh)i),
\end{eqnarray}
where $\sigma$ is the usual sigmoid function.  (I'll bet this is easy
to show, just by summing out all the other visible units.)
Furthermore, the $v_i$ are independent of one another, given $h$.
This too would be easy to show --- it'll be a straightforward
consequence of the bipartite nature of the graph. So we can compute
the probability of $v$, given $h$, simply by multiplying sigmoids.

Let's suppose we wanted to train an RBM with a set of images.  The
images would correspond to the visible units, while the hidden units
would be feature detectors.  The idea is to adjust the weights and
biases so that training images have a high probability, i.e., a low
energy.  

In a little more detail, suppose we input a training image.  Then we
can stochastically pick a corresponding value for the hidden units.
Now, feed that back, and stochastically choose a value for the image.
In an ideal world, we'd recover the original image.  We modify the
weights in such a way as to improve the fidelity of the recovered
image.

Well, the penny finally drops: an RBM can be viewed as a neural
network in which the transitions are probabilistic.  That's all!
Frankly, we don't even really need the stuff about ground states,
although it's a beautiful thing to keep in mind.

\textbf{Hinton and Salakhutdinov (2006):}
\link{http://scholar.google.ca/scholar?cluster=15344645275208957628}{(link)}

Their RBM uses ``symmetrically weighted connections''.  It is not
clear to me what this means.  It seems to mean that the biases are the
same on hidden and visible units.  I don't see how that can be ---
aren't there different numbers of such units?

So the idea is to take an RBM, and then use the training data to find
a new set of features.  We then use the features generated by the
training data as a \emph{new} set of training data, for another RBM.
We use that to find new features.  And so on, through multiple levels
of RBMs.  We then use backpropagation to fine-tune the whole thing.
It appears that the backpropagation is done with the weights treated
as though in a deterministic neural network, not stochastic, as in an
RBM.

In a bit more detail, when working with real-valued data, the visible
units in later RBMs were set to the activation probabilities of
previous hidden units.  I.e., probabilities became data.

H and S used a deep network with 784-400-200-100-50-25-6 units.  That
is, they reduced 784-dimensional input data to just 6 parameters.
And, visually at least, their reconstructions were very good,
significantly better than 6-parameter PCA and similar techniques.

What makes it difficult to train deep neural networks?  I must admit,
I don't really have a great answer to this question.  Can we come up
with a good \emph{a priori} reason for thinking it will be tough?
It's not obvious that it should be tougher than a shallow network with
the same number of neurons.

H and S compare to the work of Tenenbaum \emph{et al} and Roweis and
Saul, and comment: ``Unlike nonparametric methods (cites),
autoencoders give mappings in both directions between the data and
code spaces, and they can be applied to very large data sets because
both the pretraining and the fine-tuning scale linearly in time and
space with the number of training cases.''  I don't quite understand
the comment about mappings in both directions --- I thought the
earlier work provided such mappings.  Perhaps I should look closer.

\textbf{Bengio (2012):} \link{http://arxiv.org/abs/1206.5533}{(link)}

Notes that many of the recommendations haven't been proved, they're
heuristics that have emerged out of experimentation.  ``A good
indication of the need for such validation is that different
researchers and research groups do not always agree on the practice of
training neural networks''.

Claims that the optimal learning rate is usually close to the largest
learning rate that does not cause divergence of the cost function.
Heuristic: start with a large learning rate, and if the cost function
increases, start again with a training criterion that is three times
smaller.

This can be automated by keeping track of the cost from epoch to
epoch.  If the cost got \emph{larger} during an epoch, then decrease
the training rate by a factor two, say.  If the cost got
\emph{smaller}, then increase the training rate by a factor of 1.1,
say.  How well will that work?  I worry that we'll end up with a
situation where we're mostly going back and forth between the training
rate being too high, and too low, with not enough time to really learn
anything.

Larger mini-batches allow a modest increase in learning rate.  I don't
understand the details of this.  It'd be nice to have some heuristics.
Large mini-batches will certainly reduce stochastic error from the
sampling.  Is that what's going on?  Or is there some other reason?

``Because the gradient direction is not quite the right direction of
descent, there is no point in spending a lot of computation to
estimate it precisely for gradient descent.''  In other words, do
frequent rapid estimates rather than slow accurate computations.

It seems to me that it'd be helpful to keep track of training examples
with markedly different gradients.  Those are ones which we could
learn a lot from.  There's an idea here, which is to \emph{identify
  outliers} using the gradient.  We should oversample from the
outliers.  I'll bet that improves performance, if the right
oversampling rate is chosen.  I've explored this idea further below.

Bengio confirms that for large data sets, mini-batch stochastic
gradient descent is pretty much non-optional.

The use of validation data to train hyper-learners, which learn
hyper-parameters for a learning algorithm.

Comments that the initial learning rate is often the single most
important hyper-parameter.  ``If there is only time to optimize one
hyper-parameter and one uses stochastic gradient descent, then this is
the hyper-parameter that is worth tuning.''  Also comments that
there's often little benefit to doing anything other than keeping the
learning rate constant.  When doing otherwise, Bengio suggests a
strategy of keeping the learning rate constant for the first $\tau$
steps, and then decreasing it as $1/ t$, where $t$ is the number of
steps.  Note that this strategy is not the same as the (exponential)
automated strategy I describe above.  Suggests setting $\tau$ by
waiting until the cost goes up.  Also suggests setting multiple values
for the schedule, and seeing how they compare.

Mini-batch size: between 1 and a few hundreds.  Typical value of 32.
Notes that this mostly affects computation time, not the final value
of the cost.

Number of epochs: Watch the validation error, and stop once we're
beginning to overfit.

Momentum: smooth out gradient by taking an average of recent
gradients.

Comments that increasing the number of hidden neurons in all layers
results in a quadratic increase in time.  It's not clear to me why
that should be the case --- obviously there is a quadratic increase in
the number of weights, and so a quadratic increase in time per epoch.
But maybe it'll take a larger number of epochs to converge?

``[W]e found that using the same size for all layers worked generally
better or the same as using a decreasing size (pyramid-like) or
increasing size (upside down pyramid), but of course this may be data
dependent.''

I am surprised by this.  It seems to contradict our ideas about
feature learning.  It'd be good to look at Larochelle et al's results.
Perhaps it reflects the fact that \emph{more} high level concepts can
be formed out of the ``atoms'' of input than there are atoms.

``For most tasks that we worked on, find that an overcomplete first
hidden layer works better than an undercomplete one.''

It's not really clear why this is the case.  Again, it may be that
it's because there are more high-level concepts than low level one.
Still, that seems to be at odds with my intuition about autoencoders.

States that this is particularly true for unsupervised learning.  That
\emph{is} consistent with the idea that it's because there are many
different abstractions possible, far more than basic features.

Claims that there is a ``clean Bayesian justification'' for
regularization as the negative log-prior.  The discussion that follows
is extremely interesting and I'm still sorting it out.  The picture
that emerges seems to be that what we're doing when learning is using
some kind of maximum likelihood estimation.  In particular, we start
with some sort of prior in parameter space --- a Gaussian --- and then
try to find the weights maximizing the probability of the parameters
(weights), given the training data.  I need to unpack this still
further: it's regularization as a form of maximum likelihood.  For now
I'll proceed, and then return to this later.

Normalization: Claims that we should normalize the regularization
parameter by $B / T$, where $B$ is the mini-batch size, and $T$ is the
number of training examples.  This is consonant with what I've
observed.

Early stopping and L2 regularization: comments that these two are
essentially equivalent, and that one may as well drop L2 rgularization
when engaged in early stopping.  I don't believe this.  The solution
spaces will be completely different in the two cases.  I'm happy to
believe that \emph{sometimes} they'll give the same result, but see no
reason to believe that they'll always give the same outcome.

L1 regularization and feature selection: Comments that this strongly
suppresses irrelevant weights.  Also comments that you may wish to
consider doing both L1 and L2 regularization, with different
regularization parameters.  That seems sensible to me.

Q: An alternative approach to choosing $\lambda$ is to regard it as an
extra parameter beyond the weights, and to apply gradient descent to
it as well.  How well would this work?  My first instinct is to think
that it won't work --- that $\lambda$ will be driven to zero.  But
upon more reflection things are more complicated than that.  It'd be
interesting to know.

Sparsity: Increase sparsity can be compensated by a larger number of
hidden units.  A sparsity-inducing penalty can be viewed as a way of
regularizing.  Note that it's no longer so easy to view this in the
Bayesian framework.  Notes that the L1 penalty seems most natural, but
is not often used.  Try to push the (mini-batch) average to a
particular constant.

Neuron nonlinearity: Bengio notes that he's most often used the
sigmoid, the tanh, $\max(0, a)$, and the hard tanh.  Interesting
remark about the sigmoid not working well as the top layer of a deep
supervised net without unsupervised pretraining.  Apparently it's okay
for auto-encoders.

Weight initialization: Sample uniformly on
$4\sqrt{6/(\mbox{fan-in}+\mbox{fan-out})}$.  This will give us a total
length equal to roughly the number of layers.

Hyper-parameter selection as an optimization problem: points out the
dangers of overfitting your validation data.

Q: When does it make sense to say that we're overfitting?

Approach to parameter search: doing it logarithmically.

Q: Does it make sense to do gradient descent on just a subset of
weights at a time? I do wonder if that wouldn't sometimes yield better
results.  Deep learning has something of this flavour.


\textbf{LeCun (1998):}
\link{http://yann.lecun.com/exdb/publis/index.html\#lecun-98}{link}

Reviews the classic two-part architecture: a feature extraction
module, followed by a trainable classifier module.  Points out that
the real goal is to shunt as much as possible out of the feature
extraction module and into the classifier module, since the first
requires hand-engineering, while the second is (much more) automated.

Makes the remarkable claim that the difference in error between test
and training set scales as $(h/N)^\alpha$, where $h$ is a measure of
how complex a classifier we're using, $N$ is the number of training
examples, and $0.5 < \alpha < 1$.  In other words, the error grows as
the complexity of the machine grows.  And it shrinks as the number of
training samples grows.  I wonder why this is the case?  Could we come
up with a model that more or less proves that this is the case?  Maybe
a renormalization argument?

``The fact that local minima do not seem to be a problem for
multi-layer neural networks is somewhat of a theoretical mystery'':
This is strange.  Maybe it's the case that it's very hard to fall down
into such local minima in high dimensions? I've personally had
problems with very simple training data, but as soon as the training
data and network become at all complex, those problems seem to vanish.
This presumably means that ``most'' local minima are pretty darn good.

The \emph{segmentation} problem: the problem of cutting up a string of
characters.  Notes that a nice heuristic is to try lots and lots of
different cuts, and for each possible cut to score the cut by using
the individual character classifier: if that classifier seems to be
working well, then chances are that you have a good cut.

The authors note that existing systems are based on hand-crafted
feature extractors, but that they will not use hand-crafted features.

MNIST: constructured by combining NIST Special database 3 (SD-3) and
Special Database 1 (SD-1).  Apparently, NIST designated SD-3 as a
training set, and SD-1 as a test set.  But the two are actually very
different from on enaother.  SD-3 is a clean data set, taken from
Census Bureau employees, while SD-1 is not so good, being taken from
high-school students.  They describe some details of how MNIST was
constructed.  I'll review a few particularly striking facts.  First,
each character is size normalized, while preserving aspect ratio, and
centred.  There was also anti-aliasing going on.  So this can all be
regarded as pre-processing of features.  The database was prepared in
three forms.  One was the form I know it.  A second was a deslanted
form.  The third reduced the image resolution.

Deslanting: Idea was to compute moments of inertia, and then to
recenter things (vertically), while downsampling to 20 by 20.  As
we'll see below this significantly improves performance.

Convolutional networks: They use local receptive fields, shared
weights, and spatial sub-sampling.  ``With local receptive fields,
neurons can extract elementary visual features such as oriented edges,
end-points, corners (or similar features in other signals such as
speech spectrograms).  These features are then combined by the
subsequent layers in order to detect higher-order features.''
``... elementary feature detectors that are useful on one part of the
image are likely to be useful across the entire image.  This knowledge
can be applied by forcing a set of units, whose receptive fields are
located at different places on the image, to have identical weight
vectors.''

``Units in a layer are organized in planes within which all the units
share the same set of weights''.  So the basic idea is to convolve the
original inputs in some small window of the inputs.  We call this a
``feature map''.  I think Hinton later calls it a kernel(?)  We will
typically have several different feature maps.  So what we have is a
convolution stage.  FOr example, we might have a 5 by 5 feature map.
This is applied to a 5 by 5 receptive field in the input, i.e., a 5 by
5 area in the input.  Each unit has 25 inputs, and so 25 weights and a
bias.  ``all the units in a feature map share the same set of 25
weights and the same bias so they detect the same feature at all
possible locations on the input.''  ``The other feature maps in the
layer use different sets of weights and biases, thereby extracting
different types of local features.''  In LeNet-5 there are 6 feature
maps.  Note that a squashing function and bias apparently are used ---
this wasn't apparent earlier, where the focus is on the convolution.
Note that the feature map output will respect translations of the
original image.

Sub-sampling: The intuition is that exact location information is not
necessary.  ``Not only is the precise position of each of those
features [identified by the feature maps] irrelevant for identifying
the pattern, it is potentially harmful because the positions are
likely to vary for different instances of the character.''  ``A simple
way to reduce the precision with which the position of distinctive
features are encoded in a feature map is to reduce the spatial
resolution of the feature map.  This can be achieved with a so-called
sub-sampling layers [\emph{sic}] which performs a local averaging and
a sub-sampling, reducing the resolution of the feature map, and
reducing the sensititivity of the output to shifts and distortions.''
In LeNet-5 they use a sub-sampling layers, which perform a kind of
local averaging and sub-sampling.  Basically, they use six 2 by 2
features maps, one for each of the previous six feature maps.  ``Each
unit computes the \emph{average} of its four inputs, multiplies it by
a trainable coefficient, adds a trainable bias, and passes the result
through a sigmoid function''.  It's notable here that we don't have
trainable weights in the ordinary fashion.  It's also notable that
things aren't overlapping in this case, unlike the local receptive
fields.  Possibilities for this layer: blurring, local max, local min.
(Depends on parameter values).  

Architecture: ``Successive layers of convolutions and sub-sampling are
typically alternated...''  Traces the origins of the idea to Hubel and
Wiesel and to Fukushima.  It sounds as though the main new thing here
is to try it out with backprop.  The paper also describes some
previous applications of convolutional neural networks to image and
speech recognition.

``Since all the weights are learned with back-propagation,
convolutional networks can be seen as synthesizing their own feature
extractor.''  Big advantage of reducing the number of parameters: it
reduces overfitting.

LeNet-5: 7 layers, not counting the input.  32 by 32 inputs.  Note
that the characters are themselves 20 by 20 pixels centered in a 28 by
28 field.

Layer C3 (third layer, convolutional): 16 feature maps.  Each unit in
each feature map is connected to several 5 by 5 neighbourhoods are
identical locations in a subset of S2's feature maps.  ``WHy not
connect every S2 feature map to every C3 feature map?''  (1) Reduce
the number of connections; (2) Forces a break in symmetry in the
network.  My guess is that it would otherwise work, but might be
slower.  ``Different feature maps are forced to extract different
(hopefully complementary) features because they get different sets of
input.''

Layer C5: 120 feature maps.  Each unit is connected to a 5 by 5
neighbourhood on all 16 of S4's feature maps.  They state that this
amounts to a full connection between S4 and C5 --- this is true
because each feature unit is just a single unit.

They use a scaled hyperbolic tangent as the squashing function.  ``As
seen before, the squashing function used in our Convolutional Networks
is $f(a) = A \tanh(Sa)$.  Symmetric functions are believed to yield
faster convergence [i.e., learn at a faster rate], although the
learning can become extremely slow if the weights are too small.  The
cause of this problem is that in weight space the origin is a fixed
point of the learning dynamics, and, although it is a saddle point, it
is attractive in almost all directions''. It seems likely to me that
we will have a similar problem with the usual sigmoid function.  They
chose their parameters to ensure $f(\pm 1) = \pm 1$, i.e., for
convenience.  ``This particular choice of parameters is merely a
convenience, and does not affect the result.''

They initialize weights with the inverse of the fan-in, omitting the
square root that I am accustomed to use.  ``The standard deviation of
the weighted sum scales like the square root of the number of inputs
when the inputs are independent, and it scales linearly with the
number of inputs if the inputs are highly correlated.  We choose to
assume the second hypothesis since some units receive highly
correlated signals.''  The second clause in the first sentence is
simply false, since the weights are set independently of the inputs.
It's interesting that their method apparently works okay anyway, i.e.,
it must be quite insensitive to this detail.

Final layer in the network: Euclidean Radial Basis functions (RBF),
one for each class (i.e., 10 in total), with 84 inputs.  The output is
the squared Euclidean distance between the inputs and the input
weights.  In other words, the RBF measures how close the input is to
the weights.  Fascinatingly, the initial values for these were set by
hand, based on very simple versions of ASCII characters.

``[O]utput units... must be off most of the time.  This is quite
difficult to achieve with sigmoid units.''  Not sure why. 

Learning schedule: $\eta = 0.0005$ for the first two epochs, $0.0002$
for the next three, $0.0001$ for the next three, $0.00005$ for the
next four, and $0.00001$ for the remaining epochs (up to 20, so it was
eight).

Distortions: ``When distorted data was used for training, the test
error rate dropped to 0.8 percent (from 0.95 percent without
deformation).''  It'd be nice to have a nice little library of
transformations.

Linear classifier: 12 percent error rate. When deslanted, gets 8.4
percent error rate.  ``Various combinations of sigmoid units, linear
units, gradient descent learning, and learning by directly solving
linear systems gave similar results''.  ``A simple improvement of the
basic linear classifier was tested.  The idea is to train each unit of
a single-layer network to separate each class from each other class.
In other words, there are ${10 \choose 2} = 45$ units.  There is still
a need to have a final decision procedure, and they simply chose the
class which beat the largest number of other classes.  ``The error
rate on the regular test set was 7.6\%''.

Baseline nearest neighbor classifier: Using Euclidean distance between
input images.  ``On the regular test set the error rate was 5.0\%.  On
the deslanted data, the error rate was 2.5\%, with $k = 3$.''

PCA: Computes the projection of the input pattern on the 40 principal
components.  ``The 40-dimensional feature vector was used as the input
of a second degree polynomial classifer.''  ``The error on the regular
test set was 3.3\%.''

Radial basis functions: Error rate of 3.6\%.

One-hidden layer neural network: Error was 4.7\% for a network with
300 hidden units.  Interesting: this is worse than my results, even
when I'm using mean-square error (I get some improvement from using
cross-entropy).  I don't know why.  My initialization is somewhat
different.  Otherwise, I can't think of any reason.  They get a
reduction to 4.5\% for a network with 1000 hidden units(!)  They did
even bettter with distortions: 3.6\% and 3.8\%, with 300 and 1000
hidden units, respectively.  When deslanted images were used, the test
error dropped to 1.6\%, with 300 hidden units.  Raises the question of
why we don't get terrible overfitting, just on parameter counting
grounds.

Two-hidden layer neural network: ``The test error rate of a
784-300-100-10 network was 3.05\%, a much better result than the
one-hidden layer network [4.7\%], obtained using marginally more
weights and connections.''  This doesn't accord with my experience
using basic backprop.  Rather, it's like their results now match up
with mine for both a single and two-hidden layer.  (Admittedly, I do
get an improvement --- albeit more modest --- if pretraining is used).
However, I'm using both the cross-entropy and different weight
initialization.  So identical results wouldn't be expected.
Increasing the network size to 784-1000-150-10 improved things only a
tiny bit, to 2.95\%.  Training with distorted patterns improved things
to 2.5\% and 2.45\%, respectively.  

LeNet-1: A small convolutional net.  It got 1.7\% test error rate.
``The fact that a network with such a small number [2,600] of
parameters can attain such a good error rate is an indication that the
architecture is appropriate for the task.''

Boosting: This is a technique which sounds like an idea I've been
wondering about: concentrating more on training data which the network
is misclassifying.

Tangent distance classifier: This is an interesting idea.  The idea is
to consider the tangent plane near a digit image, where we're
considering a (low-dimensional) submanifold generated by distortions
and translations of the images.  ``An excellent measure of `closeness'
for character images is the distance between their tangent planes,
where the set of distortions used to generate the planes includes
translations, scaling, skewing, squeezing, rotation, and line
thickness variations''.  They use this measure of distance to run a
nearest-neighbor method classifier.  They get an error rate of 1.1\%,
which is (obviously) excellent.

Support vector machines: Depending on technique, results obtained
varied between 1.4\% and 0.8\%.

\textbf{On regularization:} I'd like to understand \emph{why} we
regularize.  Certainly, regularization results in solutions with a
small norm.  But why do we not what solutions with a larger norm?
Will something bad happen to us if we allow such solutions?

The standard argument: what's bad is that overfitting can occur.  And
thus regularization helps reduce overfitting.  It'd be nice to have an
example where overfitting actually occurs.  It's really not clear that
there \emph{should} be a problem with overfitting.  In fact, neural
networks eventually become virtually invariant under rescaling of
their weights and biases.  So it's really not clear that it should
help.

Returning to regularization, here's the standard story people tell to
explain why they regularize.  The story is that they want to avoid
high-complexity solutions, in order to avoid over-fitting.  Solutions
with smaller norms are in some sense lower complexity.  And therefore
it makes sense to look for solutions with smaller norm.  One way of
doing this is to penalize solutions with larger norms.  Thus, we
should add a term to the cost which penalizes such solutions.

Now, this is just a story.  It's not in any sense a sharp
justification.  In fact, the impact of regularization is still being
understood.  Researchers write papers where they try different
approaches to regularization, compare them to see which works better,
and try to understand why different approaches work the way the day.

When can overfitting occur?  Typically, when there are more parameters
in the model than there is training data.  What's odd about this is
that regularization doesn't really help all that much with this
problem.  It just restricts one degee of freedom.

Many different types of regularization possible.  I will just use the
most standard and obvious, which is quadratic.  Anything which
penalizes high-complexity solutions is okay.  It's really a research
topic.

Empirically: I find that regularization seems to help.  When we
regularize I get higher accuracies, by quite a bit.  I don't
understand why that is.

Maybe I'm already overfitting, and regularization is helping reduce
that problem.  It's possible: I have 20,000 or so parameters in my
model.  It'd be nice to see if this is the case.

An example of overfitting: I'll bet I can it to overfit when we use
just 50 training examples.  And I can probably more or less prove this
using cross-validation.

Look at LeCun \emph{et al}'s results: do they regularize, or not?

\textbf{Domingos (2012):} \link{http://scholar.google.ca/scholar?cluster=4404716649035182981\&hl=en\&as\_sdt=0,5}{link}

He points out that we don't have access to the function we really want
to optimize, unlike in most optimization problems.  Instead we use
training error as a proxy for test error.  That's a very interesting
and strange situation.

``Learners combine knowledge with data to grow programs.''

Overfitting has many faces: ``the bugbear of machine learning''; ``it
comes in many forms that are not immediately obvious''.
Generalization error can be decomposed into bias and variance.  Bias
is the tendency to keep learn the same wrong things.  Variance is the
tendency to learn random things.  E.g., an SVM (without kernel) may
have high bias if the data is nowhere close to linearly separable.
Cross-validation can itself start to overfit.

Intuition fails in high dimensions: I don't think this is quite right.
It would be better to say that it needs to be replaced in high
dimensions.

Theoretical guarantees are not what they seem: Points out that there
are effectively guarantees that can (with caveats) be put on
induction.  Very interesting.  It'd be good to understand this in
conjunction with the no-free lunch theorems.

Feature engineering is the key: Points out that the ``machine
learning'' part of a machine learning project may be tiny.  More time
spent gathering data, cleaning it, and figuring out good input
features.

More data beats a cleverer algorithm: ``As a rule, it pays to try the
simplest learners first''.  ``... the organizations that make the most
of machine learning are those that have in place an infrastructure
that makes experimenting with many different learners, data sources
and learning problems easy and effcient, and where there is a close
collaboration between machine learning experts and application domain
ones.''

Representable does not imply learnable: in other words, don't focus
all your attention on one representation (say, neural nets, or SVMs)
merely because there is some kind of universality theorem for them.

Correlation does not imply causation: Keep it in mind when
interpreting the results of machine learning algorithsm.

\textbf{Bottou (2012):}
\link{http://leon.bottou.org/papers/bottou-tricks-2012}{(link)}

Notes that there are theorems about the convergence time for batch
gradient descent (time is logarithmic in the eventual error), and for
second-order gradient descent.  It's really not clear how valuable
such results are; I guess it's comforting that they exist.

Notes that there are some powerful results about the convergence of
stochastic gradient descent, under conditions like $\sum \eta^2 <
\infty, \sum \eta = \infty$.  Apparently the ``Robbins-Siegmund
theorem'' helps with convergence.  The relevant paper is
\link{http://scholar.google.ca/scholar?cluster=509989913518206088\&hl=en\&as\_sdt=0,5}{here}.

Monitor both the training cost and the validation error: Suggests
periodically evaluating the validation error during training, and
stopping training when it hasn't improved after some time.

\textbf{Bourland and Kamp (1988):}
\link{http://scholar.google.com/scholar?cluster=17784424506773259343\&hl=en\&as\_sdt=0,5}{link}

Suggests removing nonlinearity in output.  Motivation: since we're
trying to recover the original input, claims that it's obviously not a
good idea to have the nonlinearity.  I don't see that this is true: if
the inputs are normalized to be between 0 and 1 then there shouldn't
be any problem.

With this constraint, the problem then is to find $w, b$ and $w', b'$
minimizing:
\begin{eqnarray}
  \sum_x \|w' \sigma(wx+b)+b'-x\|^2,
\end{eqnarray}
where the sum is over all input vectors $x$.  Let $X$ be the matrix
whose columns are the training vectors.  Abusing notation, let $b$ and
$b'$ be matrices whose columns are $b$ and $b'$, respectively.  Then
matrix whose columns are the outputs is given by $Y =
w'\sigma(wX+b)+b'$, where we apply $\sigma$ elementwise to the
matrix $wX+b$.  The quadratic loss function can then be written:
\begin{eqnarray}
  \| w'\sigma(wX+b)+b'-X\|^2,
\end{eqnarray}
where $\|\cdot\|$ is here the usual Frobenius matrix norm.

\textbf{Notes on PCA and autoencoders:} PCA is a way of simplifying
our understanding of data in high dimensions.  Think of the space of
all possible images.  There's a subset of that space which can
plausibly be taken to represent faces.  (Note that contextual clues
can also help).  How can we characterize that subspace?  Classic
example of PCA: IQ testing.  Take a large number of different tests.
Turns out that there is a common factor.  Another nice example: a
helix in 3 dimensions.  There's a major question: how to determine the
number of hidden units?

\textbf{Ciresan (2012)}: \link{http://arxiv.org/abs/1003.0358}{link}
This uses just straight-up backprop to train a neural net --- no
convolutional nets, no pretraining, just online learning with
backprop.  The main tricks are to use numerous deformed training
images, and graphics cards to speed up learning.  Apparently, Simard
et al used a single hidden layer with 800 neurons to get an accuracy
of 99.3 percent on MNIST.  (It'd be interesting to know whether they
deformed the images?)

The paper asks whether it was really true that the pre-training is
necessary?  Can't you just train for a long time?  And the answer
seems to be yes!

They train online, using slightly deformed images, and claim that this
means they can use the whole MNIST set for validation.  This seems
suspect to me --- it relies on the deformations being more or less
independent of how the network generalizes.  Let's run with it,
however.

They trained 5 networks, with 2 to 9 hidden layers each.  From 1.34 to
12.11 million free parameters.  They have a variable learning rate
that shrinks by a constant factor after each epoch, from 0.001 down to
0.000001.  This seems absolutely crucial to their success. I'm a
little surprised by the use of the constant factor decrease, since
that will bound the ``total'' (so to speak) learning distance
travered, simply because the geometric sum converges.  It seems like
you'd get better performance if you chose a learning schedule where
terms decreased more slowly, so the sum of the learning rates
diverged.  That's true of the hyperbolic function advocated by Bengio
in his 2012 paper, whose sum will diverge (albeit, only
logarithmically).  They initialized weights uniformly at random in the
range -0.05 to 0.05 --- that's close to, but not the same as, the
$1/\sqrt{\mbox fan-in}$ that I've preferred.  They use a tanh
activation function.

They used a GPU to do computations.  It apparently sped the
deformation routine up by a factor of 10, and forwardprop and backprop
by a factor of 40!  That's a big improvement.

Typical architecture: 784-1000-500-10 neurons.  They get 0.44 percent
test error.  That's pretty close to perfect.  The most complex
architectures were: 784-2500-2000-1500-1000-500-10 and 784-9 x
10000-10.  These get test errors of 0.32 and 0.43 percent,
respectively.  Interestingly, there seems be some advantages to having
non-homogeneous numbers in the layers.

Took 93 CPU seconds to deform the MNIST images.  87 of those seconds
were for the elastic distortions, so that's what they converted to the
GPU.  When doing the conversion they converted MNIST images to 29 x 29
to get a proper center, which simplifies distortion.

\textbf{Krizhevsky (2012):}
\link{http://www.cs.toronto.edu/~hinton/absps/imagenet.pdf}{link} 1.2
million images in ImageNet 2010.  1000 classes.  650,000 neurons.
Five convolutional layers.  Max pooling layers.  Three fully-connected
layers.  1000-way softmax.  Used dropout to prevent overfitting.

Past image data sets: NORB, Caltech-101/256. CIFAR-10/100.  ``Simple
recognition tasks can be solved quite well with datasets of this size,
especially if they are augmented with label-preserving
transformations.''  ``But objects in realistic settings exhibit
considerable variability, so to learn to recognize them it is
necessary to use much larger training sets''.  LabelMe: hundreds of
thousands of fully-segmented images.  ImageNet: 15 million labeled
high-res images in over 22,000 categories.

This paper: trained a very large convolutional neural net on subsets
of ImageNet used in two competitions.  Got by far the best results
ever reported on those data sets.  Removing any convolutional layer
significantly decreased performance.  ``All of our experiments suggest
that our results can be improved simply by waiting for faster GPUs and
bigger datasets to become available.''

ImageNet: 15 million images, 22,000 categories.  ILSVRC: 1000 images
in 1000 categories.  1.2 million training images, 50,000 validation
images, and 150,000 testing images.  ILSVRC-2010: test set labels are
available.  Top-5 error rate: the fraction of test images for which
the correct label is not among the five labels considered most
probable by the model.

ImageNet has variable-resolution images.  They down-sampled to 256
$\times$ 256.  They did this by rescaling the image so the shorter
side was of length 256.  Then cropped out the central 256 $\times$ 256
patch.  They also subtracted the mean activity over the training set
from each pixel.  This was the complete pre-processing.

Architecture: 8 layers.  5 convolutional.  3 fully-connected.

ReLU Nonlinearity: Instead of sigmoid function they used $f(z) =
\max(z, 0)$.  They refer to this as a \emph{rectified linear} unit.
``Deep convolutional neural networks with ReLUs train several times
faster than their equivalents with tanh units''.  I believe it is
standard wisdom that convolutional nets work better with tanh units
than sigmoid.  ``The magnitude of the effect [faster learning...]
varies with network architecture, but networks with ReLUs consistently
learn several times faster than equivalents with saturating neurons''.

Training on multiple GPUs: Done in part because the training set
wouldn't fit into a single GPU's memory.

Local response normalization: They do a local normalization step,
essentially a kind of brightness normalization.  It reduces error
rates by a little over 1 percent.

Overlapping pooling: Again, a slight improvement.

Architecture: The first convolutional layer filters the 224 by 224 by
3 image with 96 kernels of size 11 by 11 by 3.  There is a stride
distance of 4, i.e., the distance between the receptive field centers
of neighbouring neurons.  I need to understand quite a bit more about
CNNs and pooling.

Lots of overfitting: 1.2 million examples, 10 bits of info per example
(1 in 1000 classification).  But 60 million parameters.  So
overfitting is a real problem.

Data augmentation: (1) image translations and horizontal reflections.
Extracting 224 by 224 patches.  This gives them a factor 2048 more
training data.  The network makes a prediction by extracting five 224
by 224 patches and their horizontal reflections, and averaging the
predictions made by the network's softmax layer. (2) Altering the
intensities of the RGB channels in the training images.  Perform PCA
on ImageNet and use it to modify the images.  ``This scheme
approximately captures an important property of natural images,
namely, that object identity is invariant to changes in the intensity
and color of the illumination.'' 

Dropout: ``a very efficient version of model combination that only
cost about a factor of two during training''.  Set to zero the output
of each hidden neuron with probability 0.5.  Don't contribute to
forwardprop nor to backprop.  Every time the network is trained it has
a different architecture, but the architectures share weights.  ``This
technique reduces complex co-adaptation so neurons, since a neuron
cannot rely on the presence of other neurons''.  This is going to be
useful in very large networks with a relative paucity of data.
``Without dropout, our network exhibits substantial overfitting.
Dropout roughly doubles the number of iterations required to
converge.''

Used SGD with momentum.  ``We used an equal learning rate for all
layers, which we adjusted manually throughout training.  The heuristic
which we followed was to divide the learning rate by 10 when the
validation error rate stopped improving with the current learning
rate.  The learning rate was initialized at 0.01 and reduced three
times prior to termination.''   That seems like a useful heuristic.

Results: top-1 test set error rate: 37.5 percent.  top-5 test set
error rate: 17.0 percent.  That seems incredibly good, although not
human comparable.  They also report a bunch of other results: every
single one is very, very good.

\textbf{Lee (2009):} RBMs.  Two layer.  Bipartite.  Undirected.
Binary hidden units, $h$.  Binary or real-valued visible units, $v$.
A weight matrix $W$ between the two layers.  If visible units are
binary, then we define the energy:
\begin{eqnarray}
  E = v^T W h - b^T h-c^T v,
\end{eqnarray}
where $b$ are the hidden unit biases, and $c$ are the visible unit
biases.  For real-valued visible units, modify the energy by adding a
$1/2 v^2$ term.  This model is simple enough.  How should we think
about it?  The idea is to start with a given set of values for one
layer, say the visible layer.  Then sample the hidden units.  Then
sample the visible layer.  And so on, ping-ponging back and forth.

``In principle, the RBM parameters can be optimized by performing
stochastic gradient ascent on the log-likelihood of the training
data.''  The parameters to be optimized are presumably the weights and
biases.  The likelihood is the probability of the observed outcomes
(i.e., the training data), given the particular parameters.  I assume
that the idea is that the visible units are supposed to represent the
observed data.  So we want to choose the parameters of the model in
order to maximize the probability of seeing the training data in the
visible units.  Apparently contrastive divergence is a technique for
computing the gradient of the log-likelihood.

Convolutional RBM. The weights between the hidden and visible layers
are shared among all locations in an image.  What exactly does this
mean?  Suppose we have an $N_V \times N_V$ image.  Then the input
layer apparently consists of $N_V \times N_V$ binary units.  There are
$K$ groups in the hidden layer, each an $N_H \times N_H$ array of
binary units.  So there are $N_H^2 K$ total hidden units.

We index the hidden groups by $k$.  Each hidden group has a bias,
$b_k$.  All visible units share a single bias, $c$.

For any given group, $k$, we have a single set of $N_W \times N_W$
weights (the ``filter'').  $N_W \equiv N_V-N_H+1$.  The basic idea is
to filter the inputs, but translating the filter across the input
image.

I will come back to the energy function a little later.  XXX.  We can
do Gibbs sampling to generate the appropriate distributions.




\textbf{Simard (2003)}:

\textbf{Embrechts (2010)}:

\textbf{Dropout:}

\textbf{Le (2012):} \link{https://plus.google.com/u/0/+ResearchatGoogle/posts/EMyhnBetd2F}{link}

\textbf{Le et al (2012):} 9 layers.  Locally connected.  Sparse
autoencoder.  Pooling.  Local contrast normalization.  Can discover
faces without labels.

\textbf{Seide (2011):}
\link{http://research.microsoft.com/apps/pubs/default.aspx?id=153169}{link}

\textbf{Bengio (2007):} \link{http://arxiv.org/pdf/1206.5533v2.pdf}{link}

\textbf{Ranzato (2007):}

\textbf{Larochelle (2009):}

\textbf{Wolpert (XXX):} No free lunch.

\textbf{The NIPS 2012 talks:}

\textbf{Elements of statistical learning:} \link{http://www.stanford.edu/\~hastie/local.ftp/Springer/OLD//ESLII\_print4.pdf}{link}

\textbf{No more pesky learning rates:} \link{http://arxiv.org/pdf/1206.1106.pdf}{link}

\textbf{To do:} Contrastive divergence
(http://learning.cs.toronto.edu/~hinton/absps/cdmiguel.pdf and
http://www.cs.utoronto.ca/~hinton/absps/nccd.pdf ). LeCun 1998
``Efficient BackProp''.  Dropout.  Maxout. Domingos CACM12.  Andrew
Ng's 1997 paper ``Preventing overfitting of cross-validation data''.
Blumer \emph{et al} with guarantees on induction:
(http://scholar.google.ca/scholar?cluster=11895938102761137877\&hl=en\&as\_sdt=0,5).
Would be good to understand this in conjunction with no free lunch.
NIPS papers are online.  IPAM: https://www.ipam.ucla.edu/schedule.aspx?pc=gss2012

\textbf{On tricks:} Much of what seems to be going on is the discovery
of tricks (of various generality) which can be used to improve pattern
recognition performance.  There are some general heuristics: \emph{use
  symmetry} is obviously one.

\textbf{Idea: oversampling of outliers:} This is done using online
training.  The idea is to gradually identify outliers, and to up the
probability with which they are sampled.  What's a natural way of
doing this?  What's going on is that we have costs for a large number
of different training examples.  And we'd like to spend more time on
those examples where we do worse.  

Idea: We could try using the cost as an inverted energy.  SO we tend
to spend more time in states which are high-cost / low-energy.  The
idea is to associate an energy to each training example: $E = E(x)$.
Then we sample the training examples according to the Boltzmann
distribution.  For each sample we apply backprop to update the
weights.  So we get a new energy.  Then we sample again.  And so on.
We start at a high temperature, which means we spend about the same
amount of time on all training data.  But then we gradually lower the
temperature, so we're spending more time on outliers.  The problem
with this approach is that it requires us to calculate lots of
probabilities.  However, it might be possible to use an expander
construction to do some kind of local sampling which would, however,
remain rapidly mixing.

\section{Questions}

\emph{Question:} Might we do better using something apart from the
$\sigma$ function?

\emph{Question:} Could we try doing gradient descent on the squashing
function as well?  For instance, we could use a linear combination $p
\sigma_1+(1-p) \sigma_2$ of two squashing functions, and do gradient
descent on the value of $p$ to see what is the best possible squashing
function.

\emph{Question:} Is it harder to train a deep neural network with the
same number of neurons as a shallow neural network?

\emph{Question:} Can we identify outlier training data by looking for
gradient vectors that are unusual?

\end{document}
